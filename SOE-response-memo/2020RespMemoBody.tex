\documentclass[10pt,]{article}
\usepackage{lmodern}
\usepackage{amssymb,amsmath}
\usepackage{ifxetex,ifluatex}
\usepackage{fixltx2e} % provides \textsubscript
\ifnum 0\ifxetex 1\fi\ifluatex 1\fi=0 % if pdftex
  \usepackage[T1]{fontenc}
  \usepackage[utf8]{inputenc}
\else % if luatex or xelatex
  \ifxetex
    \usepackage{mathspec}
  \else
    \usepackage{fontspec}
  \fi
  \defaultfontfeatures{Ligatures=TeX,Scale=MatchLowercase}
\fi
% use upquote if available, for straight quotes in verbatim environments
\IfFileExists{upquote.sty}{\usepackage{upquote}}{}
% use microtype if available
\IfFileExists{microtype.sty}{%
\usepackage{microtype}
\UseMicrotypeSet[protrusion]{basicmath} % disable protrusion for tt fonts
}{}
\usepackage[left=2cm, right=2cm, top=2cm, bottom=3cm, footskip = .5cm]{geometry}
\usepackage{hyperref}
\hypersetup{unicode=true,
            pdfborder={0 0 0},
            breaklinks=true}
\urlstyle{same}  % don't use monospace font for urls
\usepackage{graphicx,grffile}
\makeatletter
\def\maxwidth{\ifdim\Gin@nat@width>\linewidth\linewidth\else\Gin@nat@width\fi}
\def\maxheight{\ifdim\Gin@nat@height>\textheight\textheight\else\Gin@nat@height\fi}
\makeatother
% Scale images if necessary, so that they will not overflow the page
% margins by default, and it is still possible to overwrite the defaults
% using explicit options in \includegraphics[width, height, ...]{}
\setkeys{Gin}{width=\maxwidth,height=\maxheight,keepaspectratio}
\IfFileExists{parskip.sty}{%
\usepackage{parskip}
}{% else
\setlength{\parindent}{0pt}
\setlength{\parskip}{6pt plus 2pt minus 1pt}
}
\setlength{\emergencystretch}{3em}  % prevent overfull lines
\providecommand{\tightlist}{%
  \setlength{\itemsep}{0pt}\setlength{\parskip}{0pt}}
\setcounter{secnumdepth}{0}

%%% Use protect on footnotes to avoid problems with footnotes in titles
\let\rmarkdownfootnote\footnote%
\def\footnote{\protect\rmarkdownfootnote}

%%% Change title format to be more compact
\usepackage{titling}

% Create subtitle command for use in maketitle
\providecommand{\subtitle}[1]{
  \posttitle{
    \begin{center}\large#1\end{center}
    }
}

\setlength{\droptitle}{-2em}

  \title{}
    \pretitle{\vspace{\droptitle}}
  \posttitle{}
    \author{}
    \preauthor{}\postauthor{}
    \date{}
    \predate{}\postdate{}
  
% Set up the fonts
\usepackage[urw-palatino]{mathdesign}
\usepackage[T1]{fontenc}


% Set the language for 508
\hypersetup{
  pdftitle = {title},
  pdflang = en-US}

% Add accessibility support from http://www.richschwinn.com/accessibility
\RequirePackage{accsupp}
\RequirePackage{pdfcomment}
\newcommand{\AccTool}[2]{\BeginAccSupp{method=pdfstringdef,unicode,Alt={{#1}}}\pdftooltip{{#2}}{{#1}}\EndAccSupp{}}


% Set up the headers and footers
\usepackage{graphicx}
\usepackage{fancyhdr}
\usepackage{ifthen}
\usepackage{everypage}
\usepackage{float}
\usepackage{subfig}

% Avoid struggling over figure and table float in Rmarkdown
\let\origfigure\figure
\let\endorigfigure\endfigure
\renewenvironment{figure}[1][2] {
    \expandafter\origfigure\expandafter[H]
} {
    \endorigfigure
}

\let\origtable\table
\let\endorigtable\endtable
\renewenvironment{table}[1][2] {
    \expandafter\origtable\expandafter[H]
} {
    \endorigtable
}

% First page has the large title and NOAA logo
\pagestyle{fancy}
\fancyhf{}
\setlength\headheight{40pt}
\cfoot{\thepage}

\AddEverypageHook{%
   \ifthenelse{\value{page}=1}%
     {\rhead{\includegraphics[width=40pt]{images/NOAA_logo.png} \\ \textsf{\emph{DRAFT}}}
      \lhead{\textsf{\LARGE State of the Ecosystem 2020: Response Memo}}
      }%
     {\rhead{}
      \lhead{\textsf{\emph{State of the Ecosystem 2020: Response Memo}}}
     }
}

\renewcommand{\headrulewidth}{0.4pt}
\renewcommand{\footrulewidth}{0pt}

% Make caption fonts a bit smaller
\usepackage[font={small}]{caption}


% Change section labels to san serif
\usepackage{sectsty}
\allsectionsfont{\normalfont\sffamily\bfseries}
\usepackage{booktabs}
\usepackage{longtable}
\usepackage{array}
\usepackage{multirow}
\usepackage{wrapfig}
\usepackage{float}
\usepackage{colortbl}
\usepackage{pdflscape}
\usepackage{tabu}
\usepackage{threeparttable}
\usepackage{threeparttablex}
\usepackage[normalem]{ulem}
\usepackage{makecell}
\usepackage{xcolor}

\begin{document}

\hypertarget{introduction}{%
\section{Introduction}\label{introduction}}

In the table below we summarize all comments and requests with sources.
The Progress column briefly summarizes how we responded, with a more
detailed response in the numbered Memo Section. In the Progress column,
``SOE'' indicates a change included in the report(s).

\begin{table}[!h]
\centering
\resizebox{\linewidth}{!}{
\begin{tabular}{l|l|l|l}
\hline
Request & Source & Progress & Memo Section\\
\hline
Formal response to requests & Both Councils & this response memo & Introduction\\
\hline
Consider report card like Alaska's & Both Councils & SOE summary bullets (page 1) & 1.0\\
\hline
include summary visualization & Both Councils & SOE infographics (page 2) & 2.0\\
\hline
Include uncertainty estimates for all indicators & Both Councils & SOE survey biomass uncertainty included; feedback requested for other indicators & 3.0\\
\hline
Include Downeast ME (Scotian Shelf EPU) & NEFMC & SOE survey biomass now includes most; included in other indicators & 4.0\\
\hline
Link zooplankton abundance and or community composition to fish condition & NEFMC & SOE page 2 research spotlight & 5.0\\
\hline
Ocean acidification information & Both Councils & inadequate information for indicator; existing info summarized here & 6.0\\
\hline
Gulf Stream Index/Labrador current interaction & Both Councils & SOE Labrador current and Gulf Stream indices now included in both reports & 7.0\\
\hline
Include source for PP estimates (satellite vs in situ) & NEFMC & SOE clarified that all PP estiamtes are from satellite & 8.0\\
\hline
Shellfish growth/distribution linked to climate (system productivity) & MAFMC & project with R. Mann student to start late 2020 & 9.0\\
\hline
Esturine condition relative to power plants and temp & MAFMC & inadequate resourses to address this year & 10.0\\
\hline
Frequency and occurance of warm rings & MAFMC & SOE added indicator & 11.0\\
\hline
Cold pool index & MAFMC & SOE added indicator & 12.0\\
\hline
Nutrient inputs and water quality near shore & MAFMC & summary of data from National Estuarine Research Reserve network started, example info included here & 13.0\\
\hline
Link environmental and social, economic indicators & NEFMC & SOE habitat wind overlap, page 2 conceptual model & 14.0\\
\hline
Quantitative overlap of wind area and habitat and fishing areas & MAFMC & SOE habitat wind overlap, wind overlap with fisheries for next round & 15.0\\
\hline
Include links to Social Science websites & NEFMC & SOE link included in both reports & 16.0\\
\hline
Management complexity & MAFMC & project started by summer student in 2018, needs further analysis & 17.0\\
\hline
South atlantic managed species represented in recreational indices - break out species managed by MA, SA and AS. & MAFMC & SOE revised indicator and noted change in report & 18.0\\
\hline
Add social elements from overview conceptual model to NE conceptual model & NEFMC & older general conceptual model replaced by specific links between indicators in report & 19.0\\
\hline
Avg weight of diet components by feeding group & Both Councils & stomach fullness analysis started--species level; feedback requested & 20.0\\
\hline
North Atlantic Right Whale calf production indicator & NEFMC & SOE added indicator & 21.0\\
\hline
Distinguish managed species in report & NEFMC & SOE Council managed species separated in landings figures & 22.0\\
\hline
Marine Mammal consumption & MAFMC & SOE added discussion of seal diets & 23.0\\
\hline
Small pelagic abundance & MAFMC & SOE have survey planktivore time series but would like to improve; see also SOE forage energy density & 24.0\\
\hline
Young of Year index from multiple surveys & MAFMC & SOE fish production from NEFSC trawl; feedback reqested on how to expand & 25.0\\
\hline
Biomass of sharks & MAFMC & HMS provided landings for 3 years and working on full time series, still looking for source of biomass data & 26.0\\
\hline
Diversity metric for NEFSC trawl survey & NEFMC & need to reconcile different survey vessel catchabilites or split by vessel & 27.0\\
\hline
Mean stomach weight across feeding guilds & MAFMC & stomach fullness analysis started--species level; feedback requested & 28.0\\
\hline
Ecosystem risk score & MAFMC & SOE PP required a step towards this; feedback requested for other desired analyses & 29.0\\
\hline
Inflection points for indicators & Both Councils & inadequate resources to do inflection point and threshold analysis this year & 30.0\\
\hline
\end{tabular}}
\end{table}

\hypertarget{responses-to-comments}{%
\section{Responses to comments}\label{responses-to-comments}}


\end{document}
