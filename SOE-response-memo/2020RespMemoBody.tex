\documentclass[10pt,]{article}
\usepackage{lmodern}
\usepackage{amssymb,amsmath}
\usepackage{ifxetex,ifluatex}
\usepackage{fixltx2e} % provides \textsubscript
\ifnum 0\ifxetex 1\fi\ifluatex 1\fi=0 % if pdftex
  \usepackage[T1]{fontenc}
  \usepackage[utf8]{inputenc}
\else % if luatex or xelatex
  \ifxetex
    \usepackage{mathspec}
  \else
    \usepackage{fontspec}
  \fi
  \defaultfontfeatures{Ligatures=TeX,Scale=MatchLowercase}
\fi
% use upquote if available, for straight quotes in verbatim environments
\IfFileExists{upquote.sty}{\usepackage{upquote}}{}
% use microtype if available
\IfFileExists{microtype.sty}{%
\usepackage{microtype}
\UseMicrotypeSet[protrusion]{basicmath} % disable protrusion for tt fonts
}{}
\usepackage[left=2cm, right=2cm, top=2cm, bottom=3cm, footskip = .5cm]{geometry}
\usepackage{hyperref}
\hypersetup{unicode=true,
            pdfborder={0 0 0},
            breaklinks=true}
\urlstyle{same}  % don't use monospace font for urls
\usepackage{graphicx,grffile}
\makeatletter
\def\maxwidth{\ifdim\Gin@nat@width>\linewidth\linewidth\else\Gin@nat@width\fi}
\def\maxheight{\ifdim\Gin@nat@height>\textheight\textheight\else\Gin@nat@height\fi}
\makeatother
% Scale images if necessary, so that they will not overflow the page
% margins by default, and it is still possible to overwrite the defaults
% using explicit options in \includegraphics[width, height, ...]{}
\setkeys{Gin}{width=\maxwidth,height=\maxheight,keepaspectratio}
\IfFileExists{parskip.sty}{%
\usepackage{parskip}
}{% else
\setlength{\parindent}{0pt}
\setlength{\parskip}{6pt plus 2pt minus 1pt}
}
\setlength{\emergencystretch}{3em}  % prevent overfull lines
\providecommand{\tightlist}{%
  \setlength{\itemsep}{0pt}\setlength{\parskip}{0pt}}
\setcounter{secnumdepth}{0}

%%% Use protect on footnotes to avoid problems with footnotes in titles
\let\rmarkdownfootnote\footnote%
\def\footnote{\protect\rmarkdownfootnote}

%%% Change title format to be more compact
\usepackage{titling}

% Create subtitle command for use in maketitle
\providecommand{\subtitle}[1]{
  \posttitle{
    \begin{center}\large#1\end{center}
    }
}

\setlength{\droptitle}{-2em}

  \title{}
    \pretitle{\vspace{\droptitle}}
  \posttitle{}
    \author{}
    \preauthor{}\postauthor{}
    \date{}
    \predate{}\postdate{}
  
% Set up the fonts
\usepackage[urw-palatino]{mathdesign}
\usepackage[T1]{fontenc}


% Set the language for 508
\hypersetup{
  pdftitle = {title},
  pdflang = en-US}

% Add accessibility support from http://www.richschwinn.com/accessibility
\RequirePackage{accsupp}
\RequirePackage{pdfcomment}
\newcommand{\AccTool}[2]{\BeginAccSupp{method=pdfstringdef,unicode,Alt={{#1}}}\pdftooltip{{#2}}{{#1}}\EndAccSupp{}}


% Set up the headers and footers
\usepackage{graphicx}
\usepackage{fancyhdr}
\usepackage{ifthen}
\usepackage{everypage}
\usepackage{float}
\usepackage{subfig}

% Avoid struggling over figure and table float in Rmarkdown
\let\origfigure\figure
\let\endorigfigure\endfigure
\renewenvironment{figure}[1][2] {
    \expandafter\origfigure\expandafter[H]
} {
    \endorigfigure
}

\let\origtable\table
\let\endorigtable\endtable
\renewenvironment{table}[1][2] {
    \expandafter\origtable\expandafter[H]
} {
    \endorigtable
}

% First page has the large title and NOAA logo
\pagestyle{fancy}
\fancyhf{}
\setlength\headheight{40pt}
\cfoot{\thepage}

\AddEverypageHook{%
   \ifthenelse{\value{page}=1}%
     {\rhead{\includegraphics[width=40pt]{images/NOAA_logo.png} \\ \textsf{\emph{DRAFT}}}
      \lhead{\textsf{\LARGE State of the Ecosystem 2020: Response Memo}}
      }%
     {\rhead{}
      \lhead{\textsf{\emph{State of the Ecosystem 2020: Response Memo}}}
     }
}

\renewcommand{\headrulewidth}{0.4pt}
\renewcommand{\footrulewidth}{0pt}

% Make caption fonts a bit smaller
\usepackage[font={small}]{caption}


% Change section labels to san serif
\usepackage{sectsty}
\allsectionsfont{\normalfont\sffamily\bfseries}
\usepackage{booktabs}
\usepackage{longtable}
\usepackage{array}
\usepackage{multirow}
\usepackage{wrapfig}
\usepackage{float}
\usepackage{colortbl}
\usepackage{pdflscape}
\usepackage{tabu}
\usepackage{threeparttable}
\usepackage{threeparttablex}
\usepackage[normalem]{ulem}
\usepackage{makecell}
\usepackage{xcolor}

\begin{document}

\hypertarget{introduction}{%
\section{Introduction}\label{introduction}}

In the table below we summarize all comments and requests with sources.
The Progress column briefly summarizes how we responded, with a more
detailed response in the numbered Memo Section. In the Progress column,
``SOE'' indicates a change included in the report(s).

\begingroup\fontsize{9}{11}\selectfont

\begin{longtable}{>{\raggedright\arraybackslash}p{5cm}>{\raggedright\arraybackslash}p{2cm}>{\raggedright\arraybackslash}p{5cm}>{\raggedright\arraybackslash}p{2cm}}
\toprule
\textbf{Request} & \textbf{Source} & \textbf{Progress} & \textbf{Memo Section}\\
\midrule
\endfirsthead
\multicolumn{4}{@{}l}{\textit{(continued)}}\\
\toprule
\textbf{Request} & \textbf{Source} & \textbf{Progress} & \textbf{Memo Section}\\
\midrule
\endhead
\
\endfoot
\bottomrule
\endlastfoot
\rowcolor{gray!6}  Formal response to requests & Both Councils & this response memo & Introduction\\
Consider report card like Alaska's & Both Councils & SOE summary bullets (page 1) & 1\\
\rowcolor{gray!6}  Include summary visualization & Both Councils & SOE infographics (page 2) & 2\\
Include uncertainty estimates for all indicators & Both Councils & SOE survey biomass uncertainty included; feedback requested for other indicators & 3\\
\rowcolor{gray!6}  Include Downeast ME (Scotian Shelf EPU) & NEFMC & SOE survey biomass now includes most; included in other indicators & 4\\
Link zooplankton abundance and or community composition to fish condition & NEFMC & SOE page 2 research spotlight & 5\\
\rowcolor{gray!6}  Ocean acidification information & Both Councils & inadequate information for indicator; existing info summarized here & 6\\
Gulf Stream Index/Labrador current interaction & Both Councils & SOE Labrador current and Gulf Stream indices now included in both reports & 7\\
\rowcolor{gray!6}  Include source for PP estimates (satellite vs in situ) & NEFMC & SOE clarified that all PP estiamtes are from satellite & 8\\
Shellfish growth/distribution linked to climate (system productivity) & MAFMC & project with R. Mann student to start late 2020 & 9\\
\rowcolor{gray!6}  Esturine condition relative to power plants and temp & MAFMC & inadequate resourses to address this year & 10\\
Frequency and occurance of warm rings & MAFMC & SOE added indicator & 11\\
\rowcolor{gray!6}  Cold pool index & MAFMC & SOE added indicator & 12\\
Nutrient inputs and water quality near shore & MAFMC & summary of data from National Estuarine Research Reserve network started, example info included here & 13\\
\rowcolor{gray!6}  Link environmental and social, economic indicators & NEFMC & SOE habitat wind overlap, page 2 conceptual model & 14\\
Quantitative overlap of wind area and habitat and fishing areas & MAFMC & SOE habitat wind overlap, wind overlap with fisheries for next round & 15\\
\rowcolor{gray!6}  Include links to Social Science websites & NEFMC & SOE link included in both reports & 16\\
Management complexity & MAFMC & project started by summer student in 2018, needs further analysis & 17\\
\rowcolor{gray!6}  South atlantic managed species represented in recreational indices - break out species managed by MA, SA and AS. & MAFMC & SOE revised indicator and noted change in report & 18\\
Add social elements from overview conceptual model to NE conceptual model & NEFMC & older general conceptual model replaced by specific links between indicators in report & 19\\
\rowcolor{gray!6}  Avg weight of diet components by feeding group & Both Councils & stomach fullness analysis started--species level; feedback requested & 20\\
North Atlantic Right Whale calf production indicator & NEFMC & SOE added indicator & 21\\
\rowcolor{gray!6}  Distinguish managed species in report & NEFMC & SOE Council managed species separated in landings figures & 22\\
Marine Mammal consumption & MAFMC & SOE added discussion of seal diets & 23\\
\rowcolor{gray!6}  Small pelagic abundance & MAFMC & SOE have survey planktivore time series but would like to improve; see also SOE forage energy density & 24\\
Young of Year index from multiple surveys & MAFMC & SOE fish production from NEFSC trawl; feedback reqested on how to expand & 25\\
\rowcolor{gray!6}  Biomass of sharks & MAFMC & HMS provided landings for 3 years and working on full time series, still looking for source of biomass data & 26\\
Diversity metric for NEFSC trawl survey & NEFMC & need to reconcile different survey vessel catchabilites or split by vessel & 27\\
\rowcolor{gray!6}  Mean stomach weight across feeding guilds & MAFMC & stomach fullness analysis started--species level; feedback requested & 28\\
Ecosystem risk score & MAFMC & SOE PP required a step towards this; feedback requested for other desired analyses & 29\\
\rowcolor{gray!6}  Inflection points for indicators & Both Councils & inadequate resources to do inflection point and threshold analysis this year & 30\\*
\end{longtable}
\endgroup{}

\hypertarget{responses-to-comments}{%
\section{Responses to comments}\label{responses-to-comments}}

\hypertarget{section}{%
\subsection{1}\label{section}}

Both Councils asked for a summary ``report card'' similar to that used
in Alaska {[}\protect\hyperlink{ref-zador_ecosystem_2016}{1}{]}. The
first page of this year's SOE reports summarizes the key messages with
icons showing the message theme (e.g., commercial fisheries, fishing
communities, forage species, system productivity, etc). At present, we
synthesized key findings on both existing and new indicators. We welcome
suggestions for indicators that should always be tracked in this
section, and for further refinements to make this summary more useful.

\hypertarget{section-1}{%
\subsection{2}\label{section-1}}

Both Councils asked for a summary visualization. The first page of the
report uses icons developed to help visualize different report
components. The second page of this year's SOE report has both a map
visualizing the key oceanographic features mentioned in the report along
with fishing communities, and a conceptual model visualizing potential
linkages between report indicators. The conceptual model is discussed
further under point \protect\hyperlink{5}{5} below.

\hypertarget{section-2}{%
\subsection{3}\label{section-2}}

Both Councils asked for uncertainty estimates to be included with
indicators. As a first step, we included survey design-based uncertainty
estimates for all surveys where we had haul specific information (all
but the inshore ME-NH survey). Including this uncertainty led to a
different approach to the data, looking for true departures from
expected stable dynamics at the functional group level, and provided
insight into which trends were potentially noteworthy. Survey biomass
uncertainty is included in figure

We experimented with a model-based estimate of uncertainty for survey
biomass which accounts for both spatial and temporal sources (VAST). The
results are promising (Fig. \ref{fig:VASTtest}). This method can also
potentailly combine the inshore and offshore surveys into a single
analysis. If the SSCs and Councils consider this approach promising, we
will persue it further for next year.

Some indicators (e.g.~total landings) may have uncertainty which is
difficult to calculate (e.g.~based on unknown reporting errors). Many
other current indicators do not have straightforward uncertainty
calculaltions (e.g.~diversity indices, anomalies) so we welcome
suggestions from the SSC and Council to guide estimation for future
reports.

\hypertarget{section-3}{%
\subsection{4}\label{section-3}}

The NE SSC asked to include Downeast ME in future reports (as the
Scotian Shelf EPU which includes it has not been included in previous
reports).

\hypertarget{section-4}{%
\subsection{5}\label{section-4}}

\hypertarget{references}{%
\section*{References}\label{references}}
\addcontentsline{toc}{section}{References}

\hypertarget{refs}{}
\leavevmode\hypertarget{ref-zador_ecosystem_2016}{}%
1. Zador SG, Holsman KK, Aydin KY, Gaichas SK. Ecosystem considerations
in Alaska: The value of qualitative assessments. ICES Journal of Marine
Science: Journal du Conseil. 2016; fsw144.
doi:\href{https://doi.org/10.1093/icesjms/fsw144}{10.1093/icesjms/fsw144}


\end{document}
